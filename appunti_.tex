\documentclass{article}

\usepackage{Sweave}
\begin{document}

\subsection{Basic permutation test}

Let $X$ be data taking values in a sample space $\mathcal{X}$. Let $G$ be a finite set of transformations $g : \mathcal{X} \rightarrow \mathcal{X}$, such that $G$ is a group with respect to the operation of composition of transformations. This means that $G$ satisfies the following three properties: $G$ contains an identity element (the map $x \mapsto x$); every element of $G$ has an inverse in $G$; for all $a_1, a_2 \in G$, $a_1 \circ a_2 \in G$. This assumption of a group structure for $G$ is fundamental throughout the paper, since it ensures that $Gg = G$ for all $g \in G$, i.e. that the set $G$ is permutation invariant.

Considering a general group of transformations rather than only permutations is useful, since in many practical situations the group consists of, for example, rotations \citep{Langsrud2005, Solari2014} or maps that multiply part of the data by $-1$ \citep{Pesarin2010}. We write $g(X)$ as $gX$. Consider any test statistic $T : \mathcal{X} \rightarrow \mathbb{R}$. Throughout this paper, we are concerned with testing the following null hypothesis of permutation invariance.

\begin{definition}
Let $H_p$ be any null hypothesis which implies that the joint distribution of the test statistics $T(gX)$, $g \in G$, is invariant under all transformations in $G$ of $X$. That is, writing $G = \{a_1, \ldots, a_{#G}\}$, under $H_p$
\[
\left(T(a_1X), \ldots, T(a_{#G}X)\right) \overset{d}{=} \left(T(a_1gX), \ldots, T(a_{#G}gX)\right)
\]
for all $g \in G$.
\end{definition}

Note that \eqref{eq:null_hypothesis} holds in particular when for all $g \in G$
\[
X \overset{d}{=} gX.
\]

Composite null hypotheses are usually not of the form $H_p$, but for specific scenarios, properties of tests of such hypotheses can be established using results in this paper.

The most basic permutation test rejects $H_p$ when $T(X) > T^{(k)}(X)$, where
\[
T^{(1)}(X) \leq \cdots \leq T^{(#G)}(X)
\]
are the sorted test statistics $T(gX)$, $g \in G$, and $k = \lfloor(1 - \alpha)#G\rfloor$ with $\alpha \in [0, 1)$. As is known and stated in the following theorem, this test has level at most $\alpha$.

\begin{theorem}
Under $H_p$, $P\{T(X) > T^{(k)}(X)\} \leq \alpha$.
\end{theorem}

We now give two proofs: a conditioning-based approach and an approach without conditioning. Both approaches are more or less known. The conditioning-based proof is similar to that in \citet{Pesarin2015}, but the setting is more general. For each $x \in \mathcal{X}$, define $O_x$ to be the orbit of $x$, which is the set $\{gx : g \in G\} \subseteq \mathcal{X}$.

\begin{proof}[Proof of Theorem 1]
Let $A = \{x \in \mathcal{X} : T(x) > T^{(k)}(x)\}$ be the set of elements of the sample space that lead to rejection. Suppose $H_p$ holds. By the group structure, $Gg = G$ for all $g \in G$. Consequently, $T^{(k)}(gX) = T^{(k)}(X)$ for all $g \in G$. Thus, $\#{g \in G : gX \in A} = \#{g \in G : T(gX) > T^{(k)}(gX)\} = \#{g \in G : T(gX) > T^{(k)}(X)\} \leq \alpha#G$.

Endow the space of orbits with the $\sigma$-algebra that it inherits from the $\sigma$-algebra on $\mathcal{X}$. Analogously to the proof of Theorem 15.2.2 in \citet{Lehmann2005}, we obtain
\[
P(X \in A | O_X) = \frac{1}{#G}\#{g \in G : gX \in A}.
\]
By the argument above, this is bounded by $\alpha$. Hence,
\[
P(X \in A) = \mathbb{E}\{P(X \in A | O_X)\} \leq \alpha
\]
as was to be shown.
\end{proof}

We now state a different proof without conditioning. A similar proof can be found in \citet{Hoeffding1952} and \citet{Lehmann2005} (p. 634).

\begin{proof}[Alternative proof of Theorem 1]
By the group structure, $Gg = G$ for all $g \in G$. Hence, $T^{(k)}(gX) = T^{(k)}(X)$ for all $g \in G$. Let $h$ have the uniform distribution on $G$. Then, under $H_p$, the rejection probability is
\[
P\{T(X) > T^{(k)}(X)\} = P\{T(hX) > T^{(k)}(hX)\} = P\{T(hX) > T^{(k)}(X)\}.
\]
The first equality follows from the null hypothesis, and the second equality holds since $T^{(k)}(X) = T^{(k)}(hX)$. Since $h$ is uniform on $G$, the above probability equals
\[
\mathbb{E}\left[(#G)^{-1} \cdot \#{g \in G : T(gX) > T^{(k)}(X)\}\right] \leq \alpha,
\]
as was to be shown.
\end{proof}

\subsection{Permutation $p$ values}

Permutation $p$ values are $p$ values based on permutations of the data. Here we will discuss permutation $p$ values based on the full permutation group. $p$ values based on random permutations are considered in Section \ref{sec:random_p_values}.

It is essential to note that there is often no unique null distribution of $T(X)$, since $H_p$ often does not specify a unique null distribution of the data. Correspondingly, $T^{(k)}(X)$ should not be seen as the $(1 - \alpha)$-quantile of the null distribution.

When a test statistic $t$ is a function (which is not random) of the data and has a unique distribution under a hypothesis $H$, then a $p$ value in the strict sense, $P_H(t \geq t_\text{obs})$, is defined where $t_\text{obs}$ is the observed value of $t$. Since under $H_p$ $T(X)$ does not always have a unique null distribution, often there exists no $p$ value in the strict sense based on this test statistic. However, under Condition 1 the statistic
\[
D = \#\{g \in G : T(gX) \geq T(X)\}
\]
does have a unique null distribution. Thus, a $p$ value in the strict sense based on $-D$ is then defined. Denoting by $d$ the observed value


\end{document}
